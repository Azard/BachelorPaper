%# -*- coding: utf-8-unix -*-

%\bibliographystyle{sjtu2}%[此处用于每章都生产参考文献]
\chapter{绪论}
\label{chap:intro}

运行Android系统的设备数量随着移动互联网的兴起爆发式增长,2014年著名的视频网站YouTube超过一半的流量来自于移动设备[1]。根据Google公司的统计,截止2015年9月30日,在全球范围内已激活的Android设备已经达到14亿台[2],超过微软公司的Windows,苹果公司的Mac OS和iOS,是全球运行设备数量最多的操作系统。

\section{Android生态系统}

每台智能手机运行着数十到上百个应用程序,简称App。其中一部分是操作系统厂商的App,但用户每天接触的最多的是第三方的应用厂商或者独立开发者开发的,需要连接互联网的App服务。在Android生态系统中,开发者完成Android App的开发完后,需要通过诸如Google Play Store、腾讯应用宝、小米应用商店等Android应用市场发布App,应用市场会对开发者提交的App进行安全、内容和程序稳定性上的审核,通过审核后,用户再从应用市场搜索App进行下载安装和使用。

% TODO 生态系统图

\subsection{用户信息价值}

如何从用户处获得信息反馈,根据反馈信息改进App的设计与体验、调整运营方式是互联网移动开发中很重要的一个环节,因为用户体验直接影响到App的用户数量。部分用户反馈可以通过应用市场的打分和评价中体现,但这些反馈包含的信息太少,难以进行更加深入的分析获取有价值的数据,而且从应用市场得到的App反馈信息是App的开发运营团队不可控的。

\subsection{碎片化}

Android操作系统的一大问题是碎片化[3],全球各个手机厂商都有不计其数的不同型号的手机运行着不同版本的Android操作系统,这些手机的硬件参数不同,系统版本不一,但有“开放手机联盟(Open Handset Alliance)”的存在,让这些不同的手机都能运行规范的有一定兼容性的Android App。但对于软件开发者来说,所开发的软件要在这些款式各异的手机上都能正确运行时意见非常困难的事情,在App的测试过程中也不可能覆盖所有的手机型号和罕见的极端情况,因此获取到不同型号手机的用户的错误反馈对于开发者提高App的稳定性和兼容性来说都有很大的帮助。

% TODO 碎片分布图

\subsection{存在的问题}


\section{相关工具产品}

国内市场上提供用户行为分析统计的商业产品团队主要有友盟(Umeng)[4]和TalkingData[5],他们的产品对于Android App用户行为分析方面的功能包括用户活跃度、用户构成、渠道分析、页面访问路径统计、事件转化率等,主要是session统计得到的数据进行处理的结果,还有比较简单的崩溃信息收集功能,实现的效果各异,报表可制定性不强。国外市场上做用户行为分析统计的产品主要有Google Analytics for Android和Yahoo公司的Flurry Analytics[6],其中Flurry Analytics的功能丰富,可以对用户行为统计数据进行较为复杂的分析和报表输出。国内外的学术界也有对Android应用的用户使用行为进行分析的研究[7][8]。

做App崩溃信息统计分析的商业产品有Twitter公司的Fabric,该产品可以将所有用户的App崩溃信息反馈给开发团队,并且拥有一整套团队测试工具进行辅助。开源软件ACRA[9]也是做Android App崩溃信息统计的工具,因为是开源软件需要使用者自己搭建服务端,主要功能完善并且发送的信息有很强的自定义性。

