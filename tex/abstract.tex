%# -*- coding: utf-8-unix -*-
%%==================================================
%% abstract.tex for SJTU Master Thesis
%%==================================================

\begin{abstract}

随着移动互联网的兴起,运行Android系统的设备数量爆发式增长。通常每台Android系统的手机或者平板电脑都运行着数十到上百个应用程序,这些移动设备上的应用程序简称App。其中部分App属于系统相关,由系统开发商开发,但更大一部分的App是由第三方开发者提供,用户通过应用商店下载安装。

开发者从用户设备上获取App使用的反馈,根据反馈信息对App的设计进行改进,调整运营方式,提高用户体验,是移动互联网开发中的重要环节。

由于Android系统的开放式策略,Android平台的碎片化问题非常严重,包括系统版本碎片化和设备硬件型号碎片化,因此要开发出兼容多个系统版本和海量不同型号设备的App是一件具有挑战性的事情。获取App在不同设备、不同环境下的运行状况,是否崩溃,对提高用户体验也非常重要。

本篇论文介绍了Appetizer,是一套收集Android平台用户使用App行为以及App卡顿、崩溃信息的系统,包括一个能够集成在Android应用中的轻量级软件开发工具包,和具有良好可扩展性的用于接收并处理数据的服务端程序。

本文描述了Appetizer的功能和设计,并与市面上现有的类似产品进行比较。
% TODO 开销总结

\keywords{\large Android \quad 碎片化 \quad 移动设备 \quad 信息收集 \quad 用户行为}
\end{abstract}

\begin{englishabstract}

With the development of mobile networks,  the mobile Interent connectivity becomes universe, which also motivates a great surge in the availability of various mobile devices, especially Android devices initially developed by Google. Nowadays, every Android device, like smartphone and tablet runs tens or even hundres of mobile applications, also terms "App" in short. Though some these apps are prebuilt in the mobile operating system or provided by the device vendors, most of the installed apps are 3rd-party apps downloaded from online app market websites.

It is quite common that app developers wants to retrieve feedbacks about their apps from users' devices, to improve app design, adjust operation strategies and improve user experience. This feedback phase serves as an important stage in mobile app developments.

Due to the openness nature of the Android ecosystem, there exist a large number of different Android devices with slight difference from the official distribution, also termed as the "Android fragmentation problem". App developers face the challenges to develop an app that can adjust to hundreds of different device models. Whether the app would crash on certain device at certain environment is an vital information for the developers to improve app quality.

This thesis introduces Appetizer, a system that collects app runtime behavior, crash and lag information. Appetizer serves as a lightweight development kit, being integrated into developers' apps. It also has a server side units that process incoming data and render statistics for developers. The design features, implementation and comparison with various off-the-shelf commercial solutions are presented in this thesis.

\keywords{\large Android \quad Fragmentation \quad Mobile device \quad Information collect \quad User beavior}
\end{englishabstract}
