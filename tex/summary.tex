%# -*- coding: utf-8-unix -*-

\begin{summary}

本篇文章介绍了Appetizer,是一个能够收集Android平台用户使用应用程序的行为、应用程序卡顿、应用程序溃信息的系统。Appetizer包括一个能够集成在Android应用中的轻量级软件开发工具包,和一套具有良好可扩展性的用于接收并处理数据的服务端程序。

第\ref{chap:intro}章绪论介绍了Android生态系统的构成,Android生态系统中存在的系统碎片化和设备碎片化的问题,以及碎片化所带来的影响,分析了Android生态系统中开发者和用户之间存在使用体验反馈难的问题,解释用户使用信息对于Android应用程序开发和运营的价值。根据Android生态系统中存在的问题,提出了Appetizer的作用就是解决这些问题,同时罗列了相关同类产品。

第\ref{chap:background}章背景介绍了Android系统和应用的架构模型及平台特点,是理解Appetizer客户端SDK设计方案的背景知识。还介绍了Appetizer客户端SDK反馈信息的内容、原因和作用。

第\ref{chap:client}章是整篇文章的核心章节,该章描述了客户端SDK的架构、设计方案以及制定出最后设计的权衡过程,架构体现了Appetizer客户端SDK相比于同类产品更加轻量化的特点。该章着重介绍了Appetizer客户端SDK关键功能的实现方式,包括网络模块的选择,持久化队列的设计实现以及如何同时保证原子性、一致性和轻量化的,用户会话收集和崩溃信息收集的实现方式,应用程序未响应(ANR)的巧妙实现方式和黑白屏时长记录。

第\ref{chap:server}章介绍了接收Appetizer客户端SDK所收集信息的服务端,着重介绍了服务端的架构设计和方案选择过程中的权衡,分别对前台(Frontend)、后台(Backend)、数据库和前后台通信的技术方案选择设计与实现进行了详细介绍。业务部分,在时间校准上设计了一种适合Appetizer应用场景的解决方案,还介绍了用户统计的实现方法,以及对崩溃信息进行快速归类的算法。

第\ref{chap:evaluation}章对Appetizer客户端SDK进行了测试,从功能上和SDK所占用的空间大小上和同类产品进行对比,结论是Appetizer客户端SDK收集的Android应用程序使用信息覆盖面较广,而且SDK占用空间明显小于同类产品,具有轻量化的优势。还从性能角度对Appetizer客户端SDK的初始化、用户会话收集、应用程序未响应(ANR)侦测功能进行了测试实验,在多个系统版本的主流Android设备上的实验结果表明,集成Appetizer客户端SDK对Android应用程序性能影响小到可以忽略,不会影响用户体验。

整篇文章从需求背景、技术背景、设计实现、功能对比和性能测试等多个角度介绍了Appetizer,展现了集成Appetizer客户端SDK能够以较小的代价解决Android开发者的痛点,开发者可以从Appetizer服务端获取到提炼过的有价值的数据,并且Appetizer和同类产品相比在轻量化和信息收集覆盖面等方面具有一定优势。

在技术角度上,本文提出并介绍了基于Android SharedPreferences和文件系统,保证原子性和一致性的轻量级持久化队列,以及客户端SDK收集信息在服务端的时间校准方法,具有一定创新性和实用价值,可以供其他开发者进行参考。

\end{summary}
